\chapter{Conclusion}
In writing this thesis, I hope to have shown how precision photometry has been a game changer in the field of exoplanets. With \textit{K2},  the legacy of \textit{Kepler} as the discoverer of smaller planets continues. In fact, currently \textit{K2} is the only operational facility that is capable of making discoveries of small planetary system such as GJ 9827. At the same time by observing targets beyond its classical regime, \textit{K2} has shown how precision photometry can be beneficial for different divisions of astronomy such as asteroseismology, Active Galactic Nuclei or galactic astronomy. While \textit{Kepler}'s four yearlong photometric coverage will probably be unmatched for its precision, length and coverage for some decades to come, \textit{K2} expansion of the scope of precision photometry has irreversibly changed the field of astronomy.

\section{Comparative Phase Curve}
The launch of \textit{TESS}, an instrument with comparable photometric capability to \textit{Kepler} but working in different bandpass, will provide an interesting opportunity to re-observe many of the planets discovered and studied by \textit{Kepler}. Since their bandpasses are different, it will allow us to see how the radius of the planet changes with wavelength, and therefore will provide some preliminary indication of the presence of the atmosphere. For hot Jupiters, if the phase curves are observable, the reflective component and the thermal component can be disentangled from these two different photometric time series as has been described in \citet{placek2016}, thereby opening new avenues of research. However, \textit{TESS} will not be observing the ecliptic, which means only for targets observed by \textit{Kepler} would such study be possible. But there is a plethora of interesting targets such as HAT-P-7b, and multiple \textit{Kepler} discovered planets which will provide a broad ground for such studies.

\begin{figure}[h!]
\centering
\includegraphics[width=\textwidth]{Response_Function.pdf}
\caption{\label{fig:response_function} Function showing different along with spectral of different spectral type star. The \textit{TESS} response function was kindly provided by Prof. Ben Placek, Wentworth Insititute of Technology.}
\end{figure}

Currently, atmospheric clouds pose a major challenge in characterizing the atmosphere of some of the most promising candidates such as GJ 1214 b \citep{kreidberg2014}. A comparative phase curve and photometry could be particularly helpful in finding out if the targets are as good as expected for atmospheric characterization follow-up. And with \textit{JWST} launch on the horizon, the field of exoplanet characterization is likely to make progress in leaps and bounds, particularly in making definitive progress towards studying the atmospheric composition, as well as providing better quality data to test our models of the atmospheres.

\section{Future Work}
\textit{K2} will likely continue for a few more campaigns, and its gamble for observing bright targets seems to be paying off. As recent as Campaign 16, there was a super-Earth sized planetary candidate around a bright target HD 73344 with $V$ = 6.9 \citep{yu2018}. If confirmed, this would be among the brightest target to have a planetary candidate, therefore great for atmospheric characterization follow-up. Besides, more campaigns are planned, which will keep the stream of discoveries of planets flowing. But, \textit{K2} will likely run out of fuel before or during Campaign 19, which would conclude one of the most prolific missions for exoplanet discovery. But this would mean we are still anticipating new data to analyze, and continue our hunt for planets.

Besides, the \textit{K2} data set is likely to hold many interesting targets which will be discovered as the data will be progressively combed with more optimized data processing techniques. One such avenue would be to look for serendipitously captured targets in the frame of the bright stars as was described in \autoref{chap:datareduction}. I intend to carry on this work during this summer. So far, I have only looked into a subset of serendipitously observed targets in a smaller subset, and looking into a larger dataset is likely to yield some interesting targets. In fact, in the smaller data set, I have already found a few interesting targets which I have not reported in this work. And some of the major challenges for such a study include finding the suitable aperture to extract the photometric data, which ties into the traditional data challenges.

\begin{figure}[b!]
\centering
\includegraphics[width=\textwidth]{QATAR_2b_Phasecurve.jpg}
\caption{\label{fig:qatarphasecurve} Combination of ellipsoidal variation (ELV) and Doppler beaming (DB) seen in Qatar-2b in \textit{K2} light curve as reported in \citet{dai2017_qatar}.}
\end{figure}

Additionally, the phase curve for K3-31 (see \autoref{fig:phasecurve_K2_31}) was an unexpected last minute find, and calculation using the planet-star parameters tells that the observed ellipsoidal variation of about 50 ppm which is higher than from expected ellipsoidal variation of amplitude 12 ppm (i.e. peak to peak value of 24 ppm), the discrepancy can be explained away with reflective as well as Doppler beaming effects. A similar phase curve was shown to be present  for Qatar-2b in \citet{dai2017_qatar} as shown in \autoref{fig:qatarphasecurve}. In our future work, we will be looking to answer if any  additional steps in detrending would allow to better preserve the fidelity of the phase curve signal. Also, once the phase curve pipeline is ironed out, it could potentially be used to validate some of the planetary candidates as has been done in the past \citep{mislis2012}. For this line of research, we will be particularly looking for the ellipsoidal variation among the planetary candidates, from which we intend get a rough estimate on the mass.

\begin{figure}[h!]
\centering
\includegraphics[width=\textwidth]{Fulton_gap.jpg}
\caption{\label{fig:FultonGap} An interesting gap (called Fulton's gap) is seen in the planet's occurrence rate vs their size commonly around 1.8R$_\oplus$ as seen in the figure above adopted from Fulton as seen in \citet{fulton2017}.}
\end{figure}

All of this effort in finding more planets ultimately ties into more fundamental questions about the occurrence rates of the planets, and their size distribution, which in turn helps to address even more fundamental questions about their origins. In fact, looking into the current data set already some of the interesting patterns have emerged. For instance, a gap known as Fulton gap in the planet size distribution as shown in \autoref{fig:FultonGap} is reported. This is already a major improvement on what was reported a few year ago (see \autoref{fig:planetdistribution}). Similarly, the relationship between the metallicity of the host star and the planet occurrence rates that was discussed by \citet{fischer2005} are being revisited with studies such as \citet{petigura2017}, which in turn has shown the relationship is strongest for hot Jupiters, and not as much as was expected for other categories of exoplanets. These are examples of how our findings of more planets will facilitate answering some of the deeper questions in the field of exoplanets.

\section{Prospects of the field}
We have seen rapid progress in different fronts of the planetary science. For instance, in the solar system exploration, the recent visits to the moons of giant planets  as well as asteroids have garnered a wealth of information. The recent \textit{Juno} mission, which is mapping Jupiter with an unprecedented detail, has led into interesting insights into Jupiter's atmosphere. For instance, cyclonic activities at the poles of Jupiter were seen \citep{kaspi2018, adriani2018} which showed atmosphere of Jupiters are more complex than expected. We have been modeling the atmosphere of hot-Jupiters for more than a decade now with some success, and such new observations within our solar system will definitely assist modelers to refine the atmospheric models of the hot Jupiters.

At the same time the field of planetary formation has taken a leap of its own with the recent developments of new radio tools such as Atacama Large Millimeter/submillimeter Array (\textit{ALMA}) \citep{alma2015}. ALMA has been a game changer, as for the first time we are finding the gaps in the proto-planetary disks \citep{andrews2016} and learning in great detail about the evolution of the disk into the planets \citep{hughes2018}. Thus, there has been progress in answering the puzzles of the planetary science at different stages. The easier ones will act as a stepping stone towards finding solutions of the harder ones.

Exoplanet research as a young field and growing at an accelerating pace. In the next two to three decades, it will very likely have even bigger breakthroughs. The majority of this will come as different tools are developed, or as technological innovation will allow to probe things that previously were impossible or simply not imagined about. As the saying goes, a rising tide raises all boats, the progress made in different branches in science as well as technology will impact the field of exoplanets by facilitating more in-depth research. The universe is a big jigsaw puzzle, and as more pieces fall in their right location, it will become easier to narrate the story.
