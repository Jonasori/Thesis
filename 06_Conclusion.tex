\chapter{Summary}
\label{chap:Summary}


We have presented ALMA observations tracing line emission CO(3-2), HCO$^+$(4-3), HCN(4-3), and CS(7-6) of d253-1536, a binary of young protoplanetary disks in the Orion Nebula Cluster. We use a gas model that assumes Keplerian rotation, local thermodynamic equilibrium, and hydrostatic equilibrium to develop synthetic images of the disks. We then use an affine-invariant Markov Chain Monte Carlo (MCMC) algorithm to explore parameter space and identify regions of best fit, comparing each model image to our data using a $\chi^2$ test. By fitting each line's emission, we are able to statistically characterize the disks' density and temperature structures.


We compare these results to the one other ONC proplyd that has been characterized, as well as to other massive disks in the $\rho$ Ophiuchus and Taurus clusters. We find that they are [TYPICAL/DIFFERENT].
