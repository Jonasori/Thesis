\chapter{Summary}
\label{chap:Summary}


We have presented ALMA observations tracing line emission CO(3-2), HCO$^+$(4-3), HCN(4-3), and CS(7-6) of d253-1536, a binary of young protoplanetary disks in the M43 region of the Orion Nebula Cluster. We model the \hco, HCN, and CO emission using a gas model that assumes Keplerian rotation, local thermodynamic equilibrium, and hydrostatic equilibrium to develop synthetic images of a model disk. We then use an affine-invariant Markov Chain Monte Carlo (MCMC) algorithm to explore parameter space and identify regions of best fit, comparing each model image to our data using a $\chi^2$ test. By fitting each line's emission, we are able to statistically characterize elements of the disks' chemical compositions and their temperature structures.

We find atmospheric temperatures that are somewhat atypically high relative to studies of other protoplanetary disks. Additionally, we find that in disk A, the binary's eastern disk, \hco and HCN abundances are two orders of magnitude higher than in disk B, possibly indicating that the disks formed separately, as suggested by \cite{Williams2014}, before joining into their current loose binary. While these values are deviations from most of the literature, they do align with the the results of work done by \citet{Factor2017} in fitting \hco emission from another disk in the same survey as these, using similar methods of analysis.

With temperature, density, and chemical characterizations of these disks, we may finally begin to answer the question of how unique we - the Solar System - are. Probably more justification here.




Thanks to Meredith, Kevin, Sam, others?

J.P. is funded by the Connecticut Space Grant's Undergraduate Research Fellowship, Undergraduate Scholarship, and Travel Grant, as well as Wesleyan University’s Research in the Sciences Fellowship.

This paper makes use of the following ALMA data: ADS/JAO.ALMA#2011.0.00028.S. ALMA is a partnership of ESO (representing its member states), NSF (USA) and NINS (Japan), together with NRC (Canada), MOST and ASIAA (Taiwan), and KASI (Republic of Korea), in cooperation with the Republic of Chile. The Joint ALMA Observatory is operated by ESO, AUI/NRAO and NAOJ. The National Radio Astronomy Observatory is a facility of the National Science Foundation operated under cooperative agreement by Associated Universities, Inc.

Recognition of: ALMA+
Software packages: Pandas \citep{McKinney2010,McKinney2011}, NumPy \citep{VanDerWalt2011}, emcee \citep{ForemanMackey2013}

maybe matplotlib, ipython
