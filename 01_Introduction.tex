\chapter{Introduction}

%\iffalse
Origins have long fascinated the human mind. Something else here.

To fully understand our history, we would hope to understand 1. how we, individually, became conscious; 2. before that, how we developed into human beings; 3. before that, how life began on Earth; 4. before that, how Earth and the Solar system formed; 5. before that, how the Sun formed; and 6. before that, how the Universe formed. The first and final of these questions are generally beyond the scope of traditional Western science, although some neuroscientists and cosmologists (as well as religious scholars, seekers, and philosophers) still try. The second and penultimate questions - how the species (the host of our consciousness) and the Sun (the host of our bodies) came to be - mirror one another both poetically and in that scientists seem to have a fairly solid grasp on each of them, using Darwin's theories and the Hertzprung-Russell Diagram to chart the evolution of our species and our host star. At the center of these six questions, how Life and Earth came into being, we find another symmetry, perhaps most notably thanks to the fact that it is here that science and (Abrahamic) religion disagree most strongly: to the religious mind, the Earth and Life were divinely Created, drawn out from Nothingness into their current, stable forms, while to modern science, life seems to have been the result of some fortuitous chemical mixing and our home, the Earth, the simple by-product of an inefficient stellar formation process somewhere in the suburbia of a minor galaxy in an infinitely vast Universe, both still evolving and changing.


This thesis is an exploration of a miniscule step in the modern Western science's process of finding an acceptable solution to the question of how the Earth and Solar System formed.
%\fi

Planetary systems, including our own Solar System, are born from the circumstellar disks of gas and dust that surround young stars. Protoplanetary disks, or proplyds, are young ($\leq$ 10Myr) circumstellar disks, which are characterized by their large abundance of gas, which typically outweighs the disk's dust by a factor of 100 to 1. However, as these disks are influenced by gravitational, chemical, and viscous forces, their gas dissipates and the disks settle into stabler, nearly gas-free debris disks, like our familiar local Solar System. But while we can observe with relative ease the current state of our local debris disk, understanding the process that brought us here - the nature of our proplyd, four and a half billion years ago - is more difficult. To understand this mystery, we must turn to observations of other comparable proplyds, the studies of which can give us insight into the processes and conditions necessary for planet formation.

Stars form when large molecular clouds develop a gravitational instability sufficient to lead to a runaway collapse process (reference), as the cloud shrinks by a factor of around ten million on its way down to a star (this is equivalent to shrinking a kilometer-scale structure down to mere millimeters). As the local material begins to self-gravitate, its center forms a dense core which will eventually become a young star (binaries are also common; are they more likely in HMSFRs?). However, the collapse also leads, almost inevitably (citation), to a circumstellar disk, needed to shed angular momentum from the protostar. These proplyds present flared radial structures, typically extending several hundred AU (vincente & Alves 05). Temperatures in their outer reaches are typically in the 10-100 K range; masses range from ones to tens of Jovian masses (Andrews and Williams 05). Since these disks form directly out of the collapse process, they, like their stellar host and the initial molecular cloud, are composed almost exclusively of molecular hydrogen.

After a few million years, the gas in these disks tends to deplete almost entirely, through processes including accretion onto the host star, blowing out from radiation pressure, and becoming locked up in rocky bodies. These new debris disks are made up of what is thought to be second generation dust, created by the grinding down of boulders and planetesimals, since any primordial dust from the initial collapse would likely have been blown out by this time. With a few notable exceptions, these disks have no detectable gas. For a more complete review of disk evolution, see Hughes et al (2018).






\section{The Minimum Mass Solar Nebula}
Not convinced this is useful.

The minimum-mass solar nebula (MMSN) is a rough conceptual aid used to inform astronomers about the distribution of material is required to form a planetary system (MMSN, Weidenschilling 1977). The MMSN is the radial mass profile that our own Solar System would present if the mass of each planet were, rather than being bound up in spheres, instead ground up and spread across the ring bound by the orbits of their inferior and superior neighbors. The resulting mass profile represents the minimum surface density required to form our own proplyd and thus a way to inform our comparisons of other disks to our own.

This is, of course, an extremely approximate characterization. One assumption it makes is that planets formed in their current positions. This is a statement that we know both to be false (Walsh et al. 2011; Tsiganis et al. 2005) and consequential, since planetary migration can cause disks to lose mass by pushing competing planetesimals either out of orbit or into inner regions of the disk where they may be more susceptible to accreting onto the host star. Still, while it doesn't





\section{Submillimeter Observations}

Observations in the radio regime, at submillimeter wavelengths, allow us access to the inner regions of proplyds which would be obscured by optically-thick dust at shorter wavelengths.

In radio astronomy, we may observe two types of emission. The first is continuum emission from thermally excited dust particles.

With spectroscopically-resolved observations, we may also trace emission from rotational transitions by molecules in the proplyds gas. These observations provide important information, including not only total mass but velocity dispersions and  temperature and density profiles.

Notably absent in both forms is emission from the central star, thanks to the fact that stars are extremely weak emitters in the radio regime.

However, to understand these types of observation, one must first understand the nature of the "telescope" making the observations. What follows is a brief introduction to radio interferometry, followed by a deeper explanation of continuum and line emission.



\subsection{Interferometry}
Interferometry is a clever way to extremely high resolution observations at long wavelengths without needing to use incredibly large collecting areas. Were one to naively attempt to create a conventional telescope to capture radio emission, they would quickly recall that maximum angular resolution is given by

\begin{align}
  \theta &= 1.22 \, \frac{\lambda}{D},
\end{align}

for a telescope with a single spherical aperture, where $\theta$ is the angular resolution achieved, $\lambda$ is the wavelength of the emission being observed, $D$ is the diameter of the aperture. The trouble with this becomes immediately apparent when one recalls that light in the radio regime is on the order of millimeters to centimeters, orders of magnitude longer than the hundreds of nanometers that optical sources emit at. Consequently, to achieve a resolution comparable to that of an optical telescope, one would have to increase their aperture's diameter accordingly to match the increase in $\lambda$. Some have tried this approach: the Arecibo Observatory in Puerto Rico and the Five hundred meter Aperture Spherical Telescope in China (with diameters of 300m and 500m, respectively), but both still have extremely coarse resolution: \~25$''$ for Arecibo and \~15$''$ for FAST, observing 3cm emission. Building and maintaining apertures this big is an extreme challenge, requiring mountains to be hollowed out, making this an unappealing solution.

A clever alternative is to leverage interferometry for a solution to the problem. In an interferometric system, one may reconstruct an image from the interference patterns between light received by two or more separate apertures. In this case, the maximum angular resolution becomes proportional to the maximum distance (or baseline) between any two of the apertures, which can be made almost arbitrarily large.

While this interference process is possible to do in the optical with CCDs, it is far more difficult to execute, as light must be forced to physically interact before reaching the sensor. At longer wavelengths, however, this becomes a more feasible task, made possible by heterodyne receivers which record both the amplitude of a signal (analogous to the intensity measured by a CCD), but also the phase of that signal. Thanks to this fact, two signals may be digitally correlated after being received.






\subsection{Continuum Emission}
\subsection{Line Emission}

\section{Disks in Low-Mass Star Forming Regions}
\section{Disks in High-Mass Star Forming Regions}

\section{Previous Observations of V2434 Ori}

\section{Summary of Contents}























% The End
