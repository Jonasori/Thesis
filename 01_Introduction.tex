\chapter{Introduction}

\iffalse
Circumstellar disks of gas and dust, left over from the processes of stellar formation, are cosmic nurseries for the growth of planets. In these young stellar systems, bodies ranging from small and rocky to massive and gaseous coalesce as gravitational, chemical, and viscous evolution transform gas-rich protoplanetary disks into tenuous, nearly gas-free debris disks over millions of years. While the direct observation of planets during this stage is effectively impossible with our current technology, observations of emission from rotational transitions of gas species, scattered light from micron-sized dust grains, and thermal emission from micron to millimeter sized grains provide an indispensable tracer of the processes that shape these disks. Because the genesis of planets is directly tied to the morphology of the gas and dust, studying the anatomy of a variety of young disks is essential to understanding how worlds such as our own formed. 49 Ceti, at an estimated age of 40 million years, is just one puzzle piece of many, but its gas-rich debris disk provides important clues to deciphering the final stages of disk evolution.


Circumstellar disks, the remnants of the stellar formation process, are the birthplace of planetary systems. It is here that the primordial ingredients of stellar formation evolve over millions of years, starting as a thick, gas rich protoplanetary disk before thinning out to a rockier, dustier debris disk. Indeed, this is a familiar story for us here on Earth, as the Solar System  - our own local debris disk - underwent a similar transformation four and a half billion years ago as our Sun, then just a juvenile protostar, formed out of its stellar nursery. But while parts of this history lesson - our Genesis story - are well understood, others still present significant uncertainties. By observing and understanding other protostars and circumstellar disks, we are given the chance to walk back through time and see just how we got to where we are today.




\subsection{Circumstellar Disk Formation and Evolution}

Stars form when large molecular clouds develop a gravitational instability sufficient to lead to a runaway collapse process (reference). As the local material begins to self-gravitate, its center forms a dense core which will eventually become a young star (binaries are also common; are they more likely in HMSFRs?). However, the collapse also leads - almost inevitably (citation) - to a circumstellar disk, needed to shed angular momentum from the protostar, usually with mass around 0.5 \% M$_{\text{]odot}}$. Since these disks form directly out of the collapse process, they (like their stellar host and the initial molecular cloud) are composed almost exclusively of H_2. In these early stages, gas outweighs dust by around two orders of magnitude





\section{Summary of Contents}
This thesis will primarily deal with...
\fi
