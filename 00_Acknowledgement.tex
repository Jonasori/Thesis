\chapter{Acknowlegdements}
% \chapter{I'm going on an adventure!}

I have been blessed with an intense and intensely unexpected journey through Wesleyan. When I marked down ``studio art" as my intended major on my ED Common Application to Wesleyan back in the fall of 2013, I had absolutely no idea that, more than half a decade later, I would be finishing a masters degree in astronomy with a physics undergrad degree under my belt. It goes without saying that making that transition was one that I absolutely could not have survived without the support of an incredible network of friends, family, and mentors.

I suppose it is appropriate to first acknowledge the immense role that Dr. Donald Griswold, my high school physics teacher and mentor, has had on my life trajectory and how I see the world. It was through his deep dedication to teaching and sharing that I was lucky enough to even consider taking a physics class when I got to Wesleyan, but it is also through his mentorship that I learned to contextualize what matters in life, what is worth pursuing, and what is real. It was also his voice that, to this day and hopefully for the rest of my days, rings in my head when I get too full of myself, reminding me of the absolute power of humility. His guidance in a time of life transitions has had an immense effect on me, and for that I am deeply grateful.


It does, of course, take a great deal more than inspiration to actually push yourself to do something that is wildly uncomfortable and unfamiliar; it takes role models and people ``above" you who are genuinely invested in your success. The fact that I happened to walk on to Wesleyan's co-ed ultimate frisbee team, Throw Culture, and find that, first in the context of ultimate and college life but soon in the context of physics, astronomy, coding, and eventually circumstellar disk research, in Jesse Lieman-Sifry was phenomenally lucky. I still often wonder how differently my Wesleyan career would have been had not he been there to bribe me into learning to code, teach me how to throw frisbees far instead of how to throw frisbees well, or just be an open and brutally honest listener in moments of distress. I so clearly remember him, a senior, explaining his thesis to me, a little frosh, and the sense of awe and wonder I had at hearing that someone only three years older than me could do all that. It is because of his guidance, drive, and honesty (and a little bit of competition) that I am here, now, submitting my own thesis. Jesse has been both a peer and a mentor to me, and I am lucky to have had him in both capacities.


Role models often come in the form of peers as well; in people with whom you are clustered, who are taking the same classes at the same time as you, and yet who seem to have some unique way of seeing through problems, some gift for finding solutions, or some deep passion for spending a day listening to the Grateful Dead and modeling stellar interiors. Cail Daley and Ryan Adler-Levine have been that to me: peers who I have looked up to and tried to keep up with as best I could. Cail's profound wonderment at the world and absolute need to make sense of it has been a constant reminder to me that there is always more beauty to be discovered, while Ryan's endless willingness to share his intellect, humor, and joy with those around him serves as an immediate course-corrector when I begin feeling fed up.


I will be honest and admit that, when I really joined the astronomy department in the fall of my junior year, it was not out of any particularly deep, innate love of space and its workings, but rather because the Wesleyan Astronomy Department is an incredibly special place.

One piece of it that I am particularly grateful for, tucked away underneath the Observatory's northwest corner, is Roy Kilgard. The endless hours spent chatting over the summer, the brief chats while I was playing hooky from Planetary Science Seminar, and, of course, the never ending tolerance for my nonstop stream of emails reporting technical glitches or requests has added a richness to my Wesleyan experience that is hard to articulate. Roy's candor, willingness to learn, and deep sense of commitment to making sure that our department works in spite of getting almost no ackowledgement for it are all traits I deeply admire.


Thanks, too, to Francis Starr, for my year of research and several classes with him. Francis is the embodiment of what a liberal arts science professor should be, both as a deeply competent researcher as well as an incredibly clear, thoughtful, and passionate teacher. His understanding when I left his lab to come work for Meredith was something that I didn't totally understand at the time, but which I deeply appreciated, and his Thermal Physics class in my senior spring made me remember why physics is wonderful.


This, of course, leads me to Meredith Hughes, my advisor. In my first year at Wesleyan, I got a little fed up having to listen to Jesse ramble on and on about how wonderful she was, but now that I've had the good fortune to spend five semesters and two-ish summers with her as my guide to the wild world of radio astronomy, I understand what he meant.



And finally, of course, my deep love for my family. I am blessed with parents who were willing to shoulder the burden so that I could go to my dream school.

To michael for his wisdom
to Casey for his perseverence and drive
to Aidan for his Goodness.





\titlespacing*{\chapter}{0pt}{*2}{*3}
