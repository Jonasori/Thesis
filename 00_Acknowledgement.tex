\chapter{Acknowlegdements}
% \chapter{I'm going on an adventure!}

I have been blessed with an intense and intensely unexpected journey through Wesleyan. When I marked down ``studio art" as my intended major on my ED Common Application to Wesleyan back in the fall of 2013, I had absolutely no idea that, half a decade later, I would be finishing a masters in astronomy with a physics degree under my belt. It goes without saying that making that transition was one that I absolutely could not have survived without the support of an incredible network of friends, family, and mentors.

I think it is appropriate to first acknowledge the role that Dr. Donald Griswold, my high school physics teacher and mentor, has had on my trajectory. It was through his deep dedication to teaching and sharing that I even considered taking a physics class when I got to Wesleyan, but it is also through his mentorship that I learned to contextualize what I learned in the broader scope of life. It also was, is, and always will be his voice that rings in my head when I get too full of myself, reminding me of the absolute power of humility. His guidance in a time of transitions has had an immense effect on me, for which I am deeply grateful.

% It does, of course, take a great deal more than inspiration to actually push yourself to do something that is wildly uncomfortable and unfamiliar; it takes role models and people ``above" you who are genuinely invested in your success.

I've also been lucky to have had a number of peers who have been mentors to me while at Wes. While many of them will go unnamed here, it is worth recognizing Jesse Lieman-Sifry's influence on my Wesleyan career, first through ultimate frisbee, and soon after through coding, astronomy, and the rest of life. I still often wonder how differently my time here would have been had not he been there to bribe me into learning to code, teach me how to throw frisbees far instead of how to throw frisbees well, and be an open and brutally honest listener in moments of distress. It is because of his guidance, drive, and honesty (and a little bit of competitive grit) that I am made it from that first frisbee practice to where I am now.


Sometimes, too, role models are found in people with whom you are clustered, who are taking the same classes at the same time as you, and yet who seem to have some unique way of seeing through problems, some gift for finding solutions, or some deep passion for spending a day listening to the Grateful Dead and modeling stellar interiors. Cail Daley and Ryan Adler-Levine have been that to me: peers to whom I have looked up and tried to keep up with as best I could. Cail's deep wonder at the world and absolute need to make sense of it has been a constant reminder to me that there is always more to be discovered, while Ryan's endless willingness to share his intellect, humor, and joy with those around him serves as valuable course-corrector in times of need.


% I will be honest and admit that, when I really joined the astronomy department in the fall of my junior year, it was not out of any particularly deep, innate love of space and its workings, but rather because the Wesleyan Astronomy Department is an incredibly special place.

The Wesleyan astronomy department as a whole deserves my thanks, having welcomed me with open arms my junior fall. One piece of the department that I am particularly grateful for, tucked away underneath the Observatory's northwest corner, is Roy Kilgard. The endless hours spent talking over the summer, the brief chats while I was playing hooky from Planetary Science Seminar, and, of course, the never ending tolerance for my nonstop stream of emails reporting my technical glitches has added a richness to my Wesleyan experience that is hard to articulate. Roy's candor, willingness to learn, and deep sense of commitment to making sure that our department works in spite of getting almost no ackowledgement for it are all traits I deeply admire.

% Thanks, too, to Francis Starr, for my year of research and several classes with him. Francis is the embodiment of what a liberal arts science professor should be, both as a deeply competent researcher as well as an incredibly articulate, thoughtful, and passionate teacher.

This, of course, leads me to Meredith Hughes, my advisor. In my first year at Wesleyan, I got a little fed up having to listen to Jesse ramble on and on about how wonderful she was, but now that I've had the good fortune to spend five semesters and two-ish summers with her as my guide to the wild world of radio astronomy, I understand what he meant. Meredith embodies what I would hope to see in a liberal arts science professor: an incredibly competent scholar and researcher individually, but one also deeply committed to her students and making this knowledge that we spend all day and night pursuing accessible. I have no idea how I ended up so lucky as to have her as my advisor for these last few years, but however it happened, I'm glad it did.


Finally, I thank my family. Thanks to Michael and Vera for their wisdom and willingness to let me just wander into their house for days at a time. Thanks to Casey for modeling drive and hard work and for always having a wide smile ready. Thanks to Aidan for our talks and helping me see a more grounded perspective. And thanks to Mom and Dad for everything, from driving down to Wes just for a single meal together to pontificating for 20 minutes on the best way to measure an egg. This support has meant the world to me, and without it I (obviously) wouldn't be here, now, writing this darn thing.








\titlespacing*{\chapter}{0pt}{*2}{*3}
