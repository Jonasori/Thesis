\chapter{Results}
\label{chap:results}

\section{Reduction Strategies}

% Ex footnote: \texttt{K2Phot}\footnote{\href{https://github.com/vincentvaneylen/k2photometry}{https://github.com/vincentvaneylen/k2photometry}} \citep{vaneylen2016}

\section{Reduction Pipeline}
% Ex something: \autoref{fig:Arc_Flux}



\subsection{Aperture}
% Another autoref. Maybe to a section? \autoref{sec:serendipitous_target}
% Another autoref, to a chapter. \autoref{chap:GJ9827}.


% What's the diff between citet, citep: \citet{white2017}, \citep{rizzuto2017}




\section{Planetary Signal Search}


\iffalse
%paragraph summarizing everything
Detrending algorithms with \textit{K2} are getting better with time, and the planet recovery rate are getting higher. It would be interesting to see how \textit{K2} compares with \textit{Kepler}. Understanding how different detrending algorithm fares, and what are the good setup for the knobs for each algorithm is something we will learn with time, and this can be seen as better quality of light curves are being extracted for later campaigns compared to earlier. The data challenges with \textit{K2} has helped in creating a host of new tools and techniques, whose utility is not likely to be limited to astronomy.
\fi

% Note that no closing info is needed.
