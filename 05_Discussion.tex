\chapter{Discussion}
\label{chap:discussion}

With the disks now fit, we may interpret our results. Since this project was based on the question of how environment influences protoplanetary disks, we would like to compare the best fit values (with consideration given to their associated uncertainties) to the values found by \citet{Factor2017}, to those found through line-emission observations of other systems, and to those derived from modeling. We also view our results in the context of planet formation potential, and maybe discuss some other stuff, too.

\section{Comparisons}

Here we compare our results to those from \citet{Factor2017}, whose work represents the only other line-emission characterization of a HMSFR protoplanetary disk's temperature and density profiles, and which consequently is our first line of contrast. Next, we would like to compare our results to the wider literature, looking for systemic variations. REWORK

%
\subsection{\citet{Factor2017}: The Other ONC Proplyd}
We begin by comparing to \citet{Factor2017}, in which they used similar methods of analysis to characterize another ONC proplyd, drawn from the same survey as our binary. Their results are presented in \ref{table:factor_fits}.


\begin{table}
  \centering
  \begin{threeparttable}
    \caption{Best-fit Results from \citet{Factor2017}}
    \label{table:factor_fits}
    \renewcommand{\arraystretch}{1.2}
    \begin{tabular}{l c c c c c }
      \toprule \toprule
      %\multirow{2}{*}{Parameter} & \multirow{2}{*}{Disk A}    & \multicolumn{2}{c}{Disk B} \\
      \multirow{2}{*}{Parameter} & \hco (thesis) & \hco (paper)  & HCN    & CO    & \hco \& CO \\
      \midrule %\midrule
      q                          & -0.02         & 0.17          & -0.18  & -0.33 & -0.24       \\
      T$_\text{atms}$ (\si{\K})  & 22            & 190           & 19     & 70    & 86          \\
      X$_\text{mol}$             & [-10]         & -10.07        & -6.7   & [-4]  & -10.04, [-4] \\
      \bottomrule
    \end{tabular}
    \begin{tablenotes}\footnotesize
      \item[*] CO is also values from the paper. In his thesis, these are [-0.3, 41, [-4]].
    \end{tablenotes}
  \end{threeparttable}
\end{table}

Comparing our results for disk A to theirs, we can see that our atmospheric temperature for the \hco fit is fairly well-aligned with their result. However, while our HCN fit is in quite close agreement with that of \hco, it is significantly higher than what they found.

Additionally, their temperature structures are systematically negative save for in the \hco line, where it is marginally positive. Ours, conversely, are systematically positive, and relatively larger than theirs. I don't know what to make of this\footnote{It also strikes me as strange that my temperatures are higher, since disk A's abundances for HCO+ are >two OoM higher than Sam's.}.


For the disk B fit, temperatures were notably higher across all lines, falling in the low 200's\footnote{Since the disk was not resolved, we were unable to fit for its temperature structure exponent, and thus fit it at -0.5 for all lines.}. Since


It is (maybe?) possible that this reflects the fact that our fits had different values for some of the fixed parameters. These include R$_{crit}$, which we fix at 100 and they fixed at 600 AU, as well as $z_q$ and $T_\text{mid}$, which they fit for and which together control how the temperature structure decays vertically (see \S\ref{subsection:physical_profs} for a more complete description). Since they fit multiple lines simultaneously, they were able to constrain these parameters, which is not possible in the case of single-line fitting. In a CO and \hco fit, they found values of $z_{q, 150} = 73$ AU and T$_{mid} = 24.7$ K, in comparison to our values of 29 and 19, respectively. \textit{Is this enough to make an appreciable difference? I don't know. It seems most likely to me that those drastically different R$_C$ values could explain the different q values, since if my disks are having their densities killed way earlier, then maybe they're struggling to match the flux levels further out. }

\textit{Another question about Sam's work: His paper and thesis have different values for the \hco and CO fits. He also makes no mention of the multi-line fits in his paper, which seems weird.}


Additionally, our molecular abundances for each disk are vary widely from those reported in the \cite{Factor2017} paper. In it, they report finding canonical values for the \hco line ($\log{X_{\hco}} = -10.04$) and unexpectedly high values for the HCN line ($\log{X_{HCN}} = -6.7$). However, we find that, while both disks HCN abundances and disk B's \hco abundance are all well-aligned with literature values, disk A shows an appreciably higher abundance at $\log{X_{\hco}} = -8$.

This seems consistent with the hypothesis, based on the wide separation of the binary's components, that these disks may have formed separately, in regions with different chemistries, and drifted together later. Maybe?



We can also look to the wider literture to compare our results with disks in lower-mass environments.



\subsection{Results Compared to Disks in Low-Mass Environments}

% From Sam:
% The expected value for q in a geometrically flat, optically thin disk is -0.5 while measured values vary from -0.6 to -0.3 (Dartois et al. 2003; Rosenfeld et al. 2012a,b)

% Cleeves et al. (2014) and Walsh et al. (2013) have both simulated the chemistry in protoplanetary disks with dif- ferent levels of ionization (ether from external sources or the central star). Using a simple set of initial conditions, they evolve the disk by simulating a chemical net- work consisting of 5910 and 43366 reactions and 639 and 378 species, respectively. Both studies produce a vertical HCO+ fractional abundance profile which, with increasing height, increases by several orders of magnitude, then decreases again. The fractional abundances relative to H2 peak at 109 and 106, respectively.

% Walsh et al. (2013) modeled the structure and emission from a disk surrounding a T Tauri star, being irradiated by a nearby O star and compared the gas line emission to that from an isolated disk. They found that, in general, line emission from the irradiated disk showed higher peak values due to the warmer disk. This was not the case for HCO+ which traces the cold, dense areas of the disk. Thus, the ratio of the HCN/HCO+ peak intensities can be used to roughly characterize the level of external irradiation, with HCN/HCO+ > 1 indicating an irradiated disk and HCN/HCO+ < 1 characteristic of an isolated disk.


Similar studies of protoplanetary disks in low-mass SFRs has shown profiles more similar to those found by \cite{Factor2017}. In their studies of HD163296 and TW Hya\citep{Flaherty2015, Flaherty2018}, the authors fit emission from CO and several of its isotopologues; in all cases they report negative values of $q$, ranging from -0.77 to -0.22, and atmospheric temperatures below 100 K, in apparent disagreement with our values. A similar study by \citet{Zhang2017} showed similar results, with a $q$ value of -0.47.


\citet{Miotello2017} showed using isotopologues of CO that disks in Lupus have gas/dust ratios far below the literature value of 100, instead indicating that values of order 10 or even 1 are more appropriate. This was echoed by

Compare to slide 8/11 here:
https://www.cv.nrao.edu/rocks/pdf/S2-P4_rocks2013_millar.pdf


Good overview:
https://arxiv.org/pdf/1901.10862.pdf

Williams and Best: 100:1 is too high.
https://iopscience.iop.org/article/10.1088/0004-637X/788/1/59/pdf

Really good(-looking) paper on gas/dust ratio/chem models:
https://www.aanda.org/articles/aa/pdf/2017/03/aa29556-16.pdf




\section{Planet-Forming Potential}
\label{section:fitting_procedure}

In order to gauage a disk's planet forming potential, we may begin by referring to the MMSN

One way to contextualize the results presented in \S\ref{chap:analysis} is through the lens of planet formation. This analysis traditionally begins with a comparison to the MMSN, which is the density profile that our own Solar System would have if all our planets had gas added to them until their composition matched that of the Sun, then each planet's mass was spread out in a ring along its orbital path (as discussed in \S\ref{chap:introduction}). Integrating this mass leaves us with $M_\text{MMSN} \approx 0.01 M_\odot$. It is worth reiterating that this is an extremely rough metric, build on several assumptions, and that it does not reflect \textit{minimum} mass of a planet forming potential, but rather an approximation of the mass it would take for a disk like ours to form.

With the extremely large mass of $M = 0.36M_\odot$ that we measure in the CO line, it is needless to say that the disk's mass would not be its limiting factor in planet formation.





\section{Best Fit Temperature }
\label{section:fitting_procedure}

We begin by comparing to \citet{Factor2017}, in which they used similar methods of analysis to characterize another ONC proplyd, drawn from the same survey as our binary. Their results are presented in \ref{table:factor_fits}.


\begin{table}
  \centering
  \begin{threeparttable}
    \caption{Best-fit Results from \citet{Factor2017}}
    \label{table:factor_fits}
    \renewcommand{\arraystretch}{1.2}
    \begin{tabular}{l c c c c c }
      \toprule \toprule
      %\multirow{2}{*}{Parameter} & \multirow{2}{*}{Disk A}    & \multicolumn{2}{c}{Disk B} \\
      \multirow{2}{*}{Parameter} & \hco (thesis) & \hco (paper)  & HCN    & CO    & \hco \& CO \\
      \midrule %\midrule
      q                          & -0.02         & 0.17          & -0.18  & -0.33 & -0.24       \\
      T$_\text{atms}$ (\si{\K})  & 22            & 190           & 19     & 70    & 86          \\
      X$_\text{mol}$             & [-10]         & -10.07        & -6.7   & [-4]  & -10.04, [-4] \\
      \bottomrule
    \end{tabular}
    \begin{tablenotes}\footnotesize
      \item[*] CO values are from Sams paper. In his thesis, CO vals are [-0.3, 41, [-4]].
    \end{tablenotes}
  \end{threeparttable}
\end{table}

Comparing our results for disk A to theirs, we can see that our atmospheric temperature for the \hco fit is fairly well-aligned with their result. However, while our HCN fit is in quite close agreement with that of \hco, it is significantly higher than what they found.

Additionally, their temperature structures are systematically negative save for in the \hco line, where it is marginally positive. Ours, conversely, are systematically positive, and relatively larger than theirs. I don't know what to make of this\footnote{It also strikes me as strange that my temperatures are higher, since disk A's abundances for HCO+ are >two OoM higher than Sam's.}.


For the disk B fit, temperatures were notably higher across all lines, falling in the low 200's\footnote{Since the disk was not resolved, we were unable to fit for its temperature structure exponent, and thus fit it at -0.5 for all lines.}. Since


It is (maybe?) possible that this reflects the fact that our fits had different values for some of the fixed parameters. These include R$_{crit}$, which we fix at 100 and they fixed at 600 AU, as well as $z_q$ and $T_\text{mid}$, which they fit for and which together control how the temperature structure decays vertically (see \S\ref{subsection:physical_profs} for a more complete description). Since they fit multiple lines simultaneously, they were able to constrain these parameters, which is not possible in the case of single-line fitting. In a CO and \hco fit, they found values of $z_{q, 150} = 73$ AU and T$_{mid} = 24.7$ K, in comparison to our values of 29 and 19, respectively. \textit{Is this enough to make an appreciable difference? I don't know. It seems most likely to me that those drastically different R$_C$ values could explain the different q values, since if my disks are having their densities killed way earlier, then maybe they're struggling to match the flux levels further out. }

\textit{Another question about Sam's work: His paper and thesis have different values for the \hco and CO fits. He also makes no mention of the multi-line fits in his paper, which seems weird.}


% The expected value for q in a geometrically flat, optically thin disk is -0.5 while measured values vary from -0.6 to -0.3 (Dartois et al. 2003; Rosenfeld et al. 2012a,b)






\section{HCO$^+$, HCN Abundance Structures}
\label{section:fitting_procedure}










% The End
