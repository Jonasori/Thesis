\chapter{Discussion}
\label{chap:discussion}



We find atmospheric temperatures notably higher than those found in \citet{Factor2017}, in which analysis of the HCO$^+$ line yielded an atmospheric temperature of 22K. Additionally, their negative radial temperature profile exponent (-0.22) is in disagreement with the positive value we find for Disk A here\footnote{Since disk B is unresolved, we are unable to fit its radial features.}. This is notable, since Disk A's best-fit relative chemical abundance is also two magnitudes higher than the value of -10 that disk B's settled to and that they fixed theirs at. It seems strange that these would go together (i.e. that a higher atmospheric temperature, temp str, and abundance) would all go together, since all of them increase flux. A little strange? I dunno.

This could, however, reflect the fact that our fits do not include other temperature structure elements that \citet{Factor2017} fit for, most notably $z_q$ and $T_\text{mid}$, which together control how the temperature structure decays vertically (see \S\ref{subsection:physical_profs} for a more complete description).







\section{Planet-Forming Potential}
\label{section:fitting_procedure}

In order to gauage a disk's planet forming potential, we may begin by referring to the MMSN

One way to contextualize the results presented in \S\ref{chap:analysis} is through the lens of planet formation. This analysis traditionally begins with a comparison to the MMSN, which is the density profile that our own Solar System would have if all our planets had gas added to them until their composition matched that of the Sun, then each planet's mass was spread out in a ring along its orbital path (as discussed in \S\ref{chap:introduction}). Integrating this mass leaves us with $M_\text{MMSN} \approx 0.01 M_\odot$. It is worth reiterating that this is an extremely rough metric, build on several assumptions, and that it does not reflect \textit{minimum} mass of a planet forming potential, but rather an approximation of the mass it would take for a disk like ours to form.

With the extremely large mass of $M = 0.36M_\odot$ that we measure in the CO line, it is needless to say that the disk's mass would not be its limiting factor in planet formation.


Because we did not fit for the surface density profile index, instead option to leave it fixed at $\gamma=1$ \citep{Andrews2009},

\section{Dynamical Mass of a PMS Star}
\label{section:fitting_procedure}




\section{Best Fit Temperature }
\label{section:fitting_procedure}








\section{HCO$^+$, HCN Abundance Structures}
\label{section:fitting_procedure}
